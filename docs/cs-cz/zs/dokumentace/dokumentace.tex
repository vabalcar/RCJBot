\documentclass[12pt, ngerman]{article}

\usepackage[utf8]{inputenc}
\usepackage{geometry}
\usepackage{amsmath}
\usepackage{last page}
\usepackage{fancyhdr}
\usepackage{enumitem}
\usepackage{xcolor}
\usepackage{sectsty}
\usepackage{float}
\usepackage{hyperref}
\usepackage{csquotes}

\MakeOuterQuote{"}

\newcommand{\PROGRAM}{RCJBot}
\newcommand{\PREDMET}{Java}
\newcommand{\AUTOR}{Václav Balcar}

\hypersetup{
    pdfsubject={\PREDMET},
    pdfauthor={\AUTOR},
    pdftitle={\PREDMET\ - \PROGRAM}
}

\definecolor{BackgroundColor}{HTML}{ffffff}
\definecolor{FontColor}{HTML}{152525}
\definecolor{SectionColor}{HTML}{216666}
\definecolor{SubSectionColor}{HTML}{3b8787}
\definecolor{SubSubSectionColor}{HTML}{4aa8a8}

\renewcommand{\contentsname}{Obsah}
\renewcommand{\figurename}{Obr.}

\pagecolor{BackgroundColor}
\color{FontColor}
\sectionfont{\color{SectionColor}}
\subsectionfont{\color{SubSectionColor}}
\subsubsectionfont{\color{SubSubSectionColor}}

\geometry{a4paper, total={190mm, 277mm}, left=20mm, right=20mm, top=20mm, bottom=20mm}
\setlength{\parindent}{0mm}

\title{Dokumentace programu \PROGRAM, zápočtového programu z předmětu \PREDMET}
\date{ZS 2017/2018}
\author{\AUTOR, 2. ročník, 31. studijní skupina}

\fancyhf{}
\lhead{Dokumentace}
\rhead{\PROGRAM}
\rfoot{\thepage}
\pagestyle{fancy}

\hypersetup{
    colorlinks=true,
    urlcolor=blue
}

\begin{document}
\pagenumbering{gobble}
\maketitle
\newpage
\pagenumbering{arabic}

\section{Uživatelská dokumentace}
\subsection{Účel}
RCJBot je softwarové řešení dálkového řízení a monitorování mobilního legového robota (postaveného ze stavebnice \href{https://www.lego.com/en-us/mindstorms/products/mindstorms-ev3-31313}{Lego 31313 MINDSTORMS EV3}) přes počítačovou síť pomocí  dalšího počítače.

\subsection{Ovládání}
Softwarové řešení má dvě oddělené části. První je program, který běží v řídící jednotce robota, druhá je program, který běží v počítači, pomocí kterého je robot ovládán. Oba programy jsou navrženy tak, aby byly pro uživatele přívětivé a nevyžadovaly jiné než nezbytné technické znalosti k jejich používání. Program pro počítač má GUI, program pro robota "text-based user interface" (dále TUI),což je vzhledem k hardwarovým možnostem robota ten nejpohodlnější poskytnutelný způsob interakce s programem v něm běžícím. Samotné dálkové řízení pak je realizováno pomocí stisků kláves na klávesnici řídícího počítače (mapování kláves na příkazy pro pohyb robota je konfigurovatelné). Monitoring dat z robota v počítači pak vypadá obdobně jako vizualizace dat z parkovacích senzorů v autech.

\subsection{Podporované HW komponenty robota}
Monitorovat a vizualizovat data je možné ze všech senzorů pro Lego Mindstorms EV3 i Lego Mindstorms NXT (předchozí generace Lego Mindstorms). Stejně tak je možné používat motory pro Lego Mindstorms EV3 i Lego Mindstorms NXT. Zkrátka lze používat cokoliv, co podporuje leJOS EV3.

\subsection{Požadavky}
\subsubsection{Na robota}
Program je navržen tak, aby pomocí něj bylo možné přirozeným způsobem ovládat  mobilního robota, který k pohybu využívá diferenciální, pásový nebo trojkolový podvozek (v případě trojkolky bude muset být prostředí kolo pasivní a volně se otáčející nebo ekvivalentně se chovající prvek). Dále může robot používat dálkoměr na detekci blížící se překážky, dotykový senzor pro detekci nárazu a color senzor pro detekci barvy povrchu, po němž se robot pohybuje. Data z těchto senzorů budou vizualizována výše popsaným způsobem a jiné využití těchto senzorů způsobí, že nepřirozenou vizualizaci. Ovšem je k dispozici i textová verze monitorování dat z robota, čímž je realizovaná možnost číst data ze všech podporovaných senzorů. Motory, které nejsou využity v rámci podvozku, jde ovládat samostatně a manuálně. Tudíž je možné vybudovat kromě podvozku jakoukoliv další konstrukci využívající motory a řídit ji na dálku.

\subsubsection{Na uživatele}
Na uživatele žádné zvláštní požadavky kladeny nejsou, pouze musí být schopný nainstalovat potřebný software do robota a do ovládajícího počítače.

\subsubsection{Na systém(y) uživatele}
Pro běh programu pro počítač je třeba mít na něm nainstalované JRE alespoň verze 7.\\\\
Dále musí uživatel zajistit připojení řídící jednotky robota a ovládajícího počítače ke společné počítačové síti, což přináší především požadavek na kompatibilní NIC připojitelné přes USB k robotovi. Neoficiální seznam testovaných podporovaných zařízení je následující:\\

Kompatibilní Wifi NIC:
\begin{enumerate}[leftmargin=5mm]
\item Netgear Wireless-N 150 usb adapter (WNA1100 Chipset)
\item Edimax EW-7811UN micro WiFi dongle
\end{enumerate} 

Kompatibilní Ethernet NIC:
\begin{enumerate}[leftmargin=5mm]
\item Apple USB ethernet adapter
\end{enumerate}

V robotovi je nutné mít nainstalovaný leJOS (lego Java OS), softwarové řešení portování JVM na řídící jednotky robotů Lego Mindstorms mající název leJOS. Konkrétně v tomto případě se je užita "edici" leJOSu s názvem \href{http://www.lejos.org/ev3.php}{leJOS EV3}, což je (jak název napovídá) edice leJOSu pro Lego Mindstorms EV3. Současná verze tohoto systému obsahuje JRE verze 7 a je to právě ta verze, kterou je třeba mít k běhu programu pro robota.

K instalaci leJOSu je třeba mít microSDHC paměťovou kartu s kapacitou 2-32GB. RCJBot nemá žádné zvýšené nároky na velikost perzistentní paměti v robotovi, minimální (co do kapacity) 2GB pamťová karta je plně postačující. Pro více informaci o instalaci leJOSu vizte \href{https://sourceforge.net/p/lejos/wiki/Windows Installation/}{návod na instalaci od vývojářů leJOSu}.

Dále pokud chce uživatel sestavovat a spouštět programy nebo generovat programátorskou dokumentaci pomocí předpřipravených antích skriptů, je třeba mít naistalovaný ant ve verzi alespoň 1.8.0 (ale silně doporučuji používat aktuální verzi, skripty jsou testovány antem ve verzi verzi 1.10.3).

\subsection{Jak programy sestavit}
K sestavení programů stačí otevřít kořenový adresář projektu a spustit zde příkaz "ant Export". Pak lze nalézt sestavené programy v adresáři build v kořenovém adresáři projektu. V této složce je složka pc, ve které je spustitelný jar, po jehož spuštění se otevře program pro počítač. Ve složce build je ještě složka lejos, kde se nachází program pro robota, který nelze bez robota úspěšně spustit.

\subsection{Jak programy spustit}
Ke spuštění programů stačí otevřít kořenový adresář projektu a spustit zde příkaz "ant PCRun" nebo příkaz "LejosRun", podle toho, který program se má spustit. Dále existuje příkaz "ant Run", který spustí oba programy najednou a paralelně. Před spuštěním jsou oba všechny spouštěné programy znovu sestaveny. Opětovně lze programy spouštět postupem v "Jak programy sestavit" v případě programu pro počítač nebo pomocí leJOSu zcela standardním leJOSím způsobem v případě programu pro robota.

\subsection{Jak získat programátorskou dokumentaci}
Pro získání programátorské dokumentace (javadocu) stačí otevřít kořenový adresář projektu a spustit zde příkaz "ant Javadoc". Vygenerovaná dokumentace se pak nachází v adresáři "./docs/javadoc".

\subsection{O testovacím robotovi}
Testovací robot má pásový podvozek a využívá všech senzorů ze setu Lego 31313 Mindstorms EV3. Dále má sestavené motorem ovládané dělo, které je součástí daného Lego setu. Na tomto robotovi je pak možné testovat a prezentovat všechny funkce programu RCJBot.

\subsection{O chování programu}
Můj zápočtový program je dvojicí programů, kdy jeden běží v ovládaném robotovi a druhý na řídícím počítači. Tyto dva programy spolu komunikují přes počítačovou síť. Vzhledem k tomu, že robot je popsatelný pomocí konfiguračního souboru, je zde velká volnost v konkrétní konstrukci robota, jedinou podmínkou je, že robot musí mít diferenciální podvozek. Činnost programu je následující: robot po spuštění získá potřebné informace z konfiguračního souboru, inicializuje HW a vyčká na přípojení řídícího programu. Skrz ten může uživatel po připojení k robotovi ovládat jeho pohyb i pohyb všech připojených motorů pomocí stisků kláves na klávesnici (jakých - záleží na uživateli, ovládání je přizpůsobitelné pomocí konfiguračního okna). Během dálkového řízení jsou data z vybraných sensorů (dálkoměr, touch sensor a color sensor) vizualizována v rámci vlastního swingového UI prvku. Tato vizualizace je podobná vizualizaci dat z parkovacích sensorů v autech. V případě přerušení spojení mezi robotem a řídící aplikací je o této skutečnosti uživatel informován prostřednictvím chybového okna. Následně (případně po odstranění problému) se lze znovu připojit k robotovi. Pokud se klient odpojí od robota, robot počká na připojení (dalšího) klienta. Činnost robota lze v každém okamžiku ukončit stiskem libovolného tlačítka na jeho řídící jednotce. K úplnosti dodávám, že robot a počítač spolu komunikují přes TCP a data, které si posílají, jsou zaserializované objekty. Vyhledání robota v síti je automatické, stačí znát jeho jméno (hostname).

\subsection{Ovládání programu pro robota}
Ovládání programu pro robota je popsáno výše v sekci "O chování programu".

\subsection{Konfigurace programu pro robota}
Je prováděna přes konfigurační soubory "port.conf" a "robot.conf". Tyto soubory musí být v pracovním adresáři programu a jejich formát je následující:

V souboru port.conf musí být (pouze) číslo portu, který bude použit pro čekání robota na připojení řídícího programu. Tento port musí být volný pro použití TCP protokolu.

Soubor robot.conf musí mít formát, který popisuje následující ukázkový konfigurační soubor:\\

\# konfigurační záznam jednoho sensoru má následující formát:\\
\# sensor type           mode     port readingFrequency\\
\# například:\\
sensor  EV3IRSensor    Distance S1   50\\
sensor  EV3TouchSensor Touch    S3   50\\
sensor  EV3ColorSensor ColorID  S4   50\\

\# konfigurační záznam jednoho kola podvozku a motoru, který ho řídí, mají následující formát:\\
\# wheel motorType              motorPort diameter offset gearRatio invert\\
\# například:\\
wheel  EV3LargeRegulatedMotor A         30       55     4.67      true\\
wheel  EV3LargeRegulatedMotor D         30       -55    4.67      true\\

\# konfigurační záznam ostatních motorů má následující formát:\\
\# motor motorType               motorPort\\
\# například:\\
motor  EV3MediumRegulatedMotor B\\

Tento soubor musí mít LF konce řádků.

\subsection{Ovládání programu pro počítač}
Ovládání programu pro počítač je částečně popsáno výše v sekci "O chování programu". To bych doplnil o to, že po připojení se k robotovi lze robota ovládat stisky kláves, kdy výchozí ovládání je následující:\\

jízda robota vpřed = klávesa (šipka) nahoru\\
jízda robota vzad = klávesa (šipka) dolů\\
jízda robota vlevo = klávesa (šipka) vlevo\\
jízda robota vpravo = klávesa (šipka) vpravo \\
otáčení v podvozku nepoužívaného motoru 1 vpřed = klávesa Q\\
otáčení v podvozku nepoužívaného motoru 1 vzad = klávesa A\\
otáčení v podvozku nepoužívaného motoru 2 vpřed = klávesa W\\
otáčení v podvozku nepoužívaného motoru 2 vzad = klávesa S\\
otáčení v podvozku nepoužívaného motoru 3 vpřed = klávesa E\\
otáčení v podvozku nepoužívaného motoru 3 vzad = klávesa D\\
otáčení v podvozku nepoužívaného motoru 4 vpřed = klávesa R\\
otáčení v podvozku nepoužívaného motoru 4 vzad = klávesa F\\
(očíslování v podvozku nepoužívaných motorů je stejné, jako pořadí, v jakém jsou tyto motory uvedeny v konfiguračním souboru robota)\\

a je možné toto výchozí ovládání změnit pomocí konfiguračního okna, které lze otevřít přes menu $\rightarrow$ Preferences $\rightarrow$ Controls confiuration.\\
Dále pro opětovné připojení se k robotovi je třeba znovu otevřít připojovací okno, které lze otevřít přes menu $\rightarrow$ JBot $\rightarrow$ Connect.\\
Ovládání těchto dialogových oken je intuitivní a není třeba specifického návodu na jejich používání.

\subsection{Konfigurace programu pro počítač}
Je prováděna pomocí postupů v sekci "Ovládání programu pro počítač", externí konfigurace (přes editaci konfiguračního souboru, který si program pro PC sám vytvoří během své činnosti ve svém pracovním adresáři) není doporučena.
\end{document}
